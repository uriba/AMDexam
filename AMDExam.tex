\documentclass[a4page,notitlepage]{article}
\usepackage{color,soul,amsmath,graphicx}
\usepackage[outercaption]{sidecap}
\usepackage[citestyle=authoryear]{biblatex}
\usepackage[font=small,labelfont=bf]{caption}
\sidecaptionvpos{figure}{c} \providecommand{\abs}[1]{\lvert#1\rvert}
\providecommand{\norm}[1]{\lVert#1\rVert}

\title{Algorithmic Mechanism Design, final exam} \author{Uri
  Barenholz} \date{Febuary 2014}

\begin{document}
\maketitle
\section{Answers}
\begin{enumerate}
\item We start by noting that the expected revenue of OPT is the sum
  of the expected revenue of all of the players, that is:
  \[
  \text{OPT}(S)=\sum_{j=1}^{|S|}E[\text{Revenue}_j]=\sum_{j=1}^{|S|}E[\phi_j(v_j)x_j(\vec{v})]
  \]
  (Where the last equality uses Mayerson's theorem)
\item
  \begin{enumerate}
  \item The algorithm will work as follows, to generate a maximal
    dense allocation do:
    \begin{enumerate}
    \item For each player calculate his density, $d_i=\frac{v_i(s_i)}{s_i}$, requiring $O(n,\log(h),\log(m))$.
    \item Sort the calculated densities in descending order requiring $O(n\log(n))$.
      For players with identical densities, sort them internally by increasing demand.
    \item Iterate on the sorted list, fulfiling the requirement for
      each player (allocating $s_i$ items to the $i$'th player on the sorted list), until reaching the point where not enough items
      remain to fulfil the requirement of the next player, requiring $O(n,\log(m),\log(h))$.
    \end{enumerate}
    This algorithm generates a dense allocation as each player that receives items receivs the exact number of items he is interested in, and he only receives these items if all players with higher density already had their demand satisfied.

    The allocation is \textit{maximum density} as all items (up to the demand of the next-highest density player's demand) have been allocated.
Therefore, every other dense allocation must satisfy the demand for all players up to the last player who's demand was satisfied and therefore does not have enough items left to satisfy the demand of another player in such a way that will not violate the dense requirement.
The special case of being able to satisfy the demand of another player that has identical density to the highest unfulfilled one but with a lower demand is also impossible as the sorted list internally sorted players with identical densities according to their (ascending) demands.

    To generate the welfare maximizing allocation assigning all items to a single player, calculate the welfare generated by assigning all the items to every possible player ($O(n,\log(h))$) and assign all items to the player generating the highest welfare.
    
    Finally, select the highest revenue generating alternative out of
    these two possibilities.
    \item
      Let's observe the maximum density solution generated by the algorithm.
      If it's revenue is more than half of the optimal revenue, we're done.
      Otherwise we note that it's revenue is (when the indices are sorted according to the densities): $\sum_{i=1}^kd_is_i<\frac{\text{OPT}}{2}$.
      As the density of all of the following players is smaller than, or equal to the density of the $k+1$'th player, it follows that the revenue that they can generate is bounded by $d_{k+1}m\geq \frac{\text{OPT}}{2}$.
      However, the value of the maximum density solution is at least the value of the density of the $k$'th player times the number of items that have been allocated by this (maximum density) allocation (denoted here by $m_a$), so:
      \begin{equation*}
        m_ad_{k+1}\leq m_ad_k\leq \sum_{i=1}^kd_is_i<\frac{\text{OPT}}{2}\leq d_{k+1}m\leq \text{OPT}
      \end{equation*}
      Now, if $m_a>\frac{m}{2}$ then we get a contradiction as 

      Now, lets partition the players to those with densities higher than or equal to $d_k$ and to those with densities lower than or equal to $d_{k+1}$, and lets observe how the optimal algorithm allocates the items between these two groups.
      If most items are allocated to the first group, then (as all members of it have higher density than those in the second group), it's revenue should be higher than half the optimal renevue, thus contradicting our previous observation.
      Therefore, most items must be allocated to the second group and the value of that group is higher than half the value of the optimal algorithm.
      So $m_sd_{k+1}\geq \frac{\text{OPT}}{2}$.
      It follows that in the maximum density allocation, less than $m_s$ items are allocated to the first group.

      Now, if $m_a > \frac{m}{2}$ then we get a contradiction as that means that more than half the items  
      As the algorithm was not able to allocate items to the $k+1$'th player, but it had more than half the items at its disposal to allocate, we get that $s_{k+1}>\frac{m}{2}$.
      Therefore, the revenue that will be generated by allocating all items to the $k+1$'th player will be $s_{k+1}d_{k+1}>\frac{m}{2}d_{k+1}\geq \frac{\text{OPT}}{2}$, concluding the proof (as there is a player that, when allocated all the items, generates a revenue of at least half the optimal revenue).


  \end{enumerate}
\end{enumerate}


\end{document}

